\documentclass[11pt,a4paper]{article}
\usepackage[margin=0.75in]{geometry}
\usepackage[dvipdfm]{graphicx}
\usepackage{bmpsize}
\usepackage{amsmath}  % Equations
\usepackage{amssymb}  % Equations
\usepackage{hyperref} % Hyperlinks in PDF documents
\usepackage{xcolor}   % Colours
\usepackage{booktabs} % Nice tables
\usepackage{listings} % Code listings
\usepackage[superscript,biblabel]{cite}
\usepackage{float}
\usepackage{wrapfig}
\usepackage{subfig}
\usepackage{multicol}
\usepackage{etoolbox}
\usepackage{relsize}
\usepackage[none]{hyphenat}
\patchcmd{\thebibliography}
  {\list}
  {\begin{multicols}{2}\smaller\list}
  {}
  {}
\appto{\endthebibliography}{\end{multicols}}

\setlength{\parskip}{\baselineskip}%
\setlength{\parindent}{0pt}%

%\graphicspath{ {figures/} }

\begin{document}

%%%%%%%%%%%%%%%%%%%%%%%%%%%%%%%%%%
%%%%%%% TITLE AND ABSTRACT %%%%%%%
%%%%%%%%%%%%%%%%%%%%%%%%%%%%%%%%%%

\title{\textbf{An investigation into the basis decompostition of orbits in Newtonian and Schwarzschild geometries, 29th August to 16th September 2016}}
\author{Lucy Oswald}
\date{}
\maketitle

%%%%%%%%%%%%%%%%%%%%%%%%%%%%%%%%%%%
%%%%%%%%%%%% MAIN BODY %%%%%%%%%%%%
%%%%%%%%%%%%%%%%%%%%%%%%%%%%%%%%%%%

\section{Creating the collection of orbits}

A classical orbit with a particular set of initial conditions was defined in two dimensions. The parameters describing this reference orbit, $z^{ref}$, were varied to produce a set of orbits, $z_{i}$, with slightly differing paths. These orbits were constructed by numerically integrating the Newtonian and Schwarzschild Hamiltonians to produce collections of classical and relativistic orbits. The effects of a perturbing mass performing a Newtonian circular orbit were included in the Hamiltonians describing the set, whilst the reference orbit was unpeturbed. Four parameters for the initial conditions were varied to produce the set: initial position, initial momentum, orientation of the orbit (varied by rotating the set of orbits) and start time in relation to that of the reference orbit.

\section{Generating the basis sets of the orbit differentials}

Once the collection of orbits had been produced, the orbit differentials were calculated by subtracting the reference orbit from each orbit in the set. The basis vectors describing the orbit differentials were generated by performing a linear transformation on the set. This was done to produce three sets of basis vectors: generated from all of the orbit differentials for both the classical and relativistic orbits, just the classical orbits and just the relativistic orbits respectively. These will hereafter be known as the combined ($\phi^C$), non-relativistic ($\phi^{NR}$) and relativistic ($\phi^R$) basis vectors.

In order to produce the basis vectors describing the sets, singular value decomposition (SVD) was performed on the collections of orbits. SVD rotates the data into its eigenvector basis, meaning that the new axes of the data are its eigenvectors. The eigenvectors are ranked by their strengths - the sizes of the eigenvalues - where the most important eigenvector in the basis set has the largest eigenvalue. The SVD theorem states that a nxp matrix A can be factorised in the following way:

\begin{equation}\label{eq:svd}
A = U \Sigma V^T
\end{equation}
where $U$ is a nxp orthogonal matrix of the eigenvectors of $A A^\dagger$, $\Sigma$ is a pxp diagonal matrix of the eigenvalues of $A A^\dagger$ with $\Sigma_{11} > \Sigma_{22} > ... > \Sigma_{pp}$ and $V$ is a pxp orthogonal matrix.

Calculating $U$ and rearranging equation \ref{eq:svd} results in equation \ref{eq:rearrange}:

\begin{equation}\label{eq:rearrange}
\Sigma V^T = U^\dagger A
\end{equation}

The matrix of normalised basis vectors $\phi$ is given by $V^T$, which can be extracted in its normalised form in the following way:

\begin{equation}
\phi_{ij} = V^T_{ij [normalised]} = \frac{(\Sigma V^T)_{ij}}{|(\Sigma V^T)_{ii}|}
\end{equation}

Since the eigenvalues of the data are organised in descending order in the matrix $\Sigma$, the normalised basis vectors are also ranked.

% SOURCES:
% http://rieke-server.physiol.washington.edu/People/Fred/Classes/545/PCA2.pdf
% http://web.mit.edu/be.400/www/SVD/Singular_Value_Decomposition.htm
% https://en.wikipedia.org/wiki/Singular_value_decomposition
% http://www.cs.princeton.edu/courses/archive/spr08/cos424/scribe_notes/0424.pdf

\section{Extracting the relativistic components from the orbit differentials}

The combined basis vectors $\phi^C_i$ were projected along the non-relativistic basis vector set $\phi^{NR}$ and these projections were subtracted from the combined set $\phi^C$, producing the set of relativistic components, $\psi$ (see equation \ref{eq:proj}). The relativistic basis set $\psi^{basis}$ was produced by performing singular value decomposition on $\psi$ as described above.

%Using Einstein summation convention:
%\begin{equation}
%\chi = (\phi^{NR})^{T}
%\end{equation}
%\begin{equation}
%\psi_{ij} = \phi^{C}_{ij} - \phi^{C}_{ik} \chi_{kn} \phi^{NR}_{nj}
%\end{equation}

\begin{equation}\label{eq:proj}
\psi_{ij} = \phi^{C}_{ij} - \sum\limits_{n} (\sum\limits_{k} \phi^C_{ik} \phi^{NR}_{nk}) \phi^{NR}_{nj}
\end{equation}

In order to remove noise due to the Newtonian perturber and the numerical integration, the strengths of the combined and non-relativistic basis vectors were inspected. The first few components: those with strengths considerably more than zero, were defined as the principle components of the basis set, whilst the rest were discarded. The number of principle components was varied depending on the size of the basis set, to ensure that important information was not lost and that the amount of noise included was minimised.

\section{Recreation of the original orbits from the relativistic components}

To reconstruct a relativistic orbit without the noise due to the Newtonian perturber, its orbit differential $\phi^R_i$ was projected onto the relativistic basis set $\psi^{basis}$ and the reference orbit was added back, as shown in equation \ref{eq:recon}. The resultant set of reconstructed relativistic orbits $\chi_{i}$ were then plotted.

\begin{equation}\label{eq:recon}
\chi_{i} = z^{ref} + \sum\limits_{n} (\sum\limits_{k} (z_{i} - z^{ref})_{k} \psi^{basis}_{nk})*\psi^{basis}_{in}
\end{equation}

\newpage
\section{Figures}

\begin{figure}[h]
\centerline{\includegraphics[width=200mm]{figure_1}}
\caption{The set of orbits and the orbit differentials with respect to the reference orbit. \label{fig:orbs}}
\end{figure}

\begin{figure}[h]
\centerline{\includegraphics[width=280mm]{figure_2}}
\caption{A perturbed relativistic orbit reconstructed from $\psi^{basis}$ to remove the Newtonian perturbation, displayed with the reference orbit, the perturbed relativistic orbit and how the orbit would look it it were unpeturbed. \label{fig:recon}}
\end{figure}

\begin{figure}[h]
\centerline{\includegraphics[width=400mm]{figure_3_zoom_apocentre}}
\caption{Close-up of the apocentres of the orbits in figure \ref{fig:recon}. \label{fig:apo}}
\end{figure}

\begin{figure}[h]
\centerline{\includegraphics[width=300mm]{figure_4_zoom_pericentre}}
\caption{Close-up of the pericentres of the orbits in figure \ref{fig:recon}. \label{fig:peri}}
\end{figure}


%%%%%%%%%%%%%%%%%%%%%%%%%%%%%%%%%%%%%%%%%%%%%%%%%%%%%%

%\section{notes}


%Figure showing the reference orbit, the peturbed rel orbit, the psi projected rel orbit all on the same plot. Aspect ratio equal. Two with different numbers of pcs hence wiggles. 

%N.B. reason that you get swirls is because 2 body interaction with perturber. Initial conditions means the phase of the interaction pushes the test particle in the wrong direction. Test out with perturber at right angles to rather than in front of BH and see if it changes! 
%Impressively this is all lost when you do the projection onto the psi basis set.


%\cite{CITATION}
%\ref{SECTION OR FIGURE}

%\begin{wrapfigure}{r}{0.4\textwidth}
%\centerline{\includegraphics[width=0.2\textwidth]{PICTURE}}
%\caption{CAPTION\cite{CITATION} \label{fig:LABEL}}
%\end{wrapfigure}

%\begin{figure}[h]
%\centerline{\includegraphics[width=80mm]{FIGURE}}
%\caption{CAPTION\cite{CITATION} \label{fig:LABEL}}
%\end{figure}

%%%%%%%%%%%%%%%%%%%%%%%%%%%%%%%%%%%
%%%%%%%%%%%%%BIBLIOGRAPHY%%%%%%%%%%%%%%%
%%%%%%%%%%%%%%%%%%%%%%%%%%%%%%%%%%%

%\bibliographystyle{unsrt}
%{\footnotesize
%\bibliography{BIBLIOGRAPHY LOCATION}}

%%%%%%%%%%%%%%%%%%%%%%%%%%%%%%%%%%%
%%%%%%%%%%%%%%APPENDIX%%%%%%%%%%%%%%%%
%%%%%%%%%%%%%%%%%%%%%%%%%%%%%%%%%%%

%\appendix

%\section{Some extra material}

%%%%%%%%%%%%%%%%%%%%%%%%%%%%%%%%%%%

\end{document}

